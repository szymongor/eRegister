\documentclass[12pt, titlepage]{article}
\usepackage{polski}
\usepackage[utf8]{inputenc}
\usepackage{amsmath}
\usepackage{amsfonts}
\usepackage{graphicx}


\begin{document}
\begin{titlepage}

\newcommand{\HRule}{\rule{\linewidth}{0.5mm}} % Defines a new command for the horizontal lines, change thickness here

\center 

\textsc{\LARGE Politechnika Śląska}\\[1.5cm] 

\textsc{\Large Wydział Matematyki Stosowanej}\\ 
\textsc{\Large Informatyka, sem. VI}\\[1cm]

\textsc{\large Aplikacje bazodanowe}\\
\textsc{\large Dokumentacja Projektu}\\[0.7cm] 


\HRule \\[0.7cm]
{ \huge \bfseries E-register}\\[0.4cm] 
\HRule \\[1.5cm]
 
\begin{minipage}{0.8\textwidth}
\begin{flushleft} \large
\emph{Autorzy:}\\
Karolina \textsc{Chrząszcz}\\
Szymon \textsc{Górnioczek}\\
Wiktor \textsc{Gruszczyński}\\
Jarosław \textsc{Kania}\\
Tomasz \textsc{Kryg}\\
\end{flushleft}
\end{minipage}\\[1cm]

\includegraphics[scale=0.5]{img/logo.jpg}\\[1cm]
{\large \today}\\[0.5cm] 

\vfill % Fill the rest of the page with whitespace

\end{titlepage}


\section{Opis ogólny wymagań}

Główną funkcjonalnością tworzonej aplikacji bazodanowej jest dziennik internetowy. Musi on przede wszystkim pozwalać na zapis oraz wgląd do ocen ucznia, ale również mieć pełną bazę użytkowników, grup, zajęć oraz przedmiotów.
Najważniejszym elementem bazy danych jest ,,użytkownik". Użytkownikiem dziennika może zostać nauczyciel, uczeń oraz opiekun. W każdym przypadku, dane tej osoby są trzymane w bazie przez dłuższy czas, nawet po opuszczeniu placówki. W drugim przypadku jedynie blokowane jest ich konto użytkownika, a dane osoby pozostają bez zmian.

\subsection{Uczeń}

Uczeń jest bez wątpienia podstawowym użytkownikiem bazy – bez niego, dziennik byłby zbędny. Jego prawa dotyczące dostępu są jednak w pewien sposób okrojone. Każdy uczeń posiadający konto w bazie, ma jedynie wgląd do swoich ocen, zajęć w których uczestniczą oraz grup, do których przynależą. Może on jedynie przeglądać informacje – nie ma prawa modyfikacji.
Uczniowie należą do pewnych grup. Grupa jest to większa ilość uczniów uczestniczących w pewnych konkretnych zajęciach. Zajęcia zawsze dotyczą jednego z możliwych przedmiotów. Z każdych zajęć, uczeń otrzymuje oceny. Podział ocen jest następujący:
\begin{itemize}
\item ocena częściowa -- pomniejsza ocena związana z pewną aktywnością ucznia na zajęciach. Może to być ocena za sprawdzian, odpowiedź lub zadanie domowe. Każda ocena częściowa ma swoją wagę, która wpływa na ważniejsze oceny opisane poniżej.
\item ocena semestralna -- ocena występująca raz na cały semestr. Jest ona wyliczana na podstawie ocen częściowych ucznia osobno dla każdego przedmiotu.
\item ocena końcowa -- główna ocena wystawiana z każdego przedmiotu po całym roku zajęć. Jest to średnia z dwóch ocen semestralnych: za pierwszy oraz za drugi semestr.
\end{itemize}
Uczeń z przedmiotu może uzyskać między innymi ocenę niedostateczną. Oznacza ona, że nie uzyskuje on promocji do następnej klasy. W takim wypadku zostaje on jedynie przepisany do innej klasy (rok wcześniej) i kontynuuje swoją naukę w tej klasie.
Dodatkowo każdy z uczniów, poza grupą należy także do klasy profilowanej. Są to klasy prowadzone przez nauczycieli i są one odpowiednio ukierunkowane. Podsumowując uczeń nie tylko należy do pewnej klasy, ale także ma osobne zajęcia w pewnych konkretnych grupach (przykładem mogą być zajęcia wychowania fizycznego, gdzie istnieją osobne grupy dla chłopców i dziewcząt).

\subsection{Nauczyciel}

Osobą pełniącą główną pieczę nad ocenami jest nauczyciel. Jest on przypisany do każdej z klas profilowanych jako jej wychowawca. Dodatkowo każdy z nauczycieli prowadzi zajecia z pewnego przedmiotu, zatem jest to osoba posiadająca nieco więcej praw od ucznia – posiada prawo wystawiania ocen. Nauczyciel samodzielnie wystawia każdą z ocen na podstawie osiągnięć uczniów z danego przedmiotu, który on sam prowadzi. Pod uwagę została również wzięta kwestia nauczyciela będącego jednocześnie rodzicem dziecka.

\subsection{Opiekun}

Osobnym użytkownikiem, mającym jedynie wgląd do ocen jest opiekun. Jest to osoba będąca zwykle rodzicem ucznia, mającego swoje konto w bazie. Zatem bez ucznia nie ma opiekuna. Opiekun jest poniekąd użytkownikiem-duchem. Może przeglądać jedynie oceny swojego podopiecznego, ale sam nie ma w bazie żadnych danych dodatkowych poza tożsamością oraz adresem zamieszkania. Jest on swego rodzaju obserwatorem, którego rolą jest dowiadywanie się o postępach w nauce swojego dziecka. 

\end{document}
